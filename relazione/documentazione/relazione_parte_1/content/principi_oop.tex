% File: principi_oop.tex
% Created: 2014-12-04
% Author: Tesser Paolo
% Email: p.tesser921@gmail.com
% 
%
% Modification History
% Version	Modifier Date	Author			Change
% ====================================================================
% 0.0.1	2014-12-04	Tesser-Paolo		inserita sezione 
% ====================================================================
% 0.0.2	2014-12-06	Tesser Paolo		inizio stesura capitolo e sottocapitoli
%

\section{Principi di Programmazione ad Oggetti}
In questa capitolo vengono descritte le scelte effettuate per implementare i principi della programmazione ad oggeti, in particolare quelli di incapsulamento e di information hiding. \\

	\subsection{Principio di modularità}
E' stato deciso di dividere la struttura dell'applicativo in diversi package. \\
Questo permette una migliore manutenzione del codice in quanto in ogni package sono contenute delle classi logicamente correlate e niente altro che possa essere di interferenza con il loro funzionamento. \\
Si è deciso di rendere inoltre il codice il più possibile estendibile e ciò è stato realizzato tramite l'utilizzo di interfacce e di classi astratte. \\
Maggiori dettagli su come è stato realizzato il tutto sono illustrati nel capitolo precedente, dedicato apposta alle scelte architetturali.

	\subsection{Principio di incapsulamento}
Per avere un buon incapsulamento si è cercato di rendere le classi il più isolate possibili, permettendone l'accesso ai suoi membri solo attraverso degli opportuni metodi getter/setter e rendendo tali membri privati. \\
Questi marcatori di accesso inoltre sono quasi tutti firmati \textbf{final} per non consentire delle future ridefinizioni da sottoclassi che ne possano compromettere il naturale comportamento. \\
TO DO
	\subsection{Principio di information hiding}
TO DO

	\subsection{Principio di sostituizione}
Per garantire che l'utente possa scrivere codice basato sulla specifica TO DO
