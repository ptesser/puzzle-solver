% File: algoritmo_1.tex
% Created: 2014-12-05
% Author: Tesser Paolo
% Email: p.tesser921@gmail.com
% 
%
% Modification History
% Version	Modifier Date	Author			Change
% ====================================================================
% 0.0.1	2014-12-05	Tesser-Paolo		inserita sezione 
% ====================================================================
% 
%

\section{Algoritmo di risoluzione}
L'algoritmo scelto per risolvere il puzzle è sequenziale, come richiesto dalla specifica di progetto. \\
Per arrivare alla soluzione vengono utilizzati due strutture dati come membri della classe PuzzleCharacter. \\
La prima struttura è la collezione HashMap, nella quale salverò in ordine casuale i tasselli ricevuti in input dal file di testo. Memorizzerò dunque l'id del tassello come chiave mentre come valore salverò l'intero pezzo (Tile). \\
La seconda struttura dati è un array bidimensionale di oggetti Tile. TO DO\\
Di seguito vengono esposte le seguenze che vengo eseguite, correlate da dei grafici che mostrano come esso agisca sulle strutture dati utilizzate.

	\begin{enumerate}
		\item \textbf{Ricerco il primo elemento del puzzle (quello in alto a sinistra).} \\
Scorro quindi una sola volta la tavola hash per cercare il tassello che ha id nord e id ovest uguale alla stringa VUOTO. \\
Una volta trovato salvo il pezzo trovato nella prima posizione dell'array bidimensionale.

		\item \textbf{Ordino la colonna più a sinistra (quella con i tasselli aventi id ovest uguale alla stringa VUOTO).} \\
TO DO
		\item \textbf{Ordino tutte le righe.} \\
TO DO
	\end{enumerate}


