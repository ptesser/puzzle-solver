% File: robustezza.tex
% Created: 2015-01-22
% Author: Tesser Paolo
% Email: p.tesser921@gmail.com
% 
%
% Modification History
% Version	Modifier Date	Author			Change
% ====================================================================
% 0.0.1		2015-01-22		Tesser Paolo	inserita sezione capitolo
% ====================================================================
%

\section{Robustezza}
Nel programma la robustezza è stata gestita in maniera approssimata. \\
Di seguito però viene illustrato ciò che è stato fatto, utile per ampliare la gestione di eventi che possono verificarsi nella rete o in caso di caduta o del server o del client. \\
In particolare non sono stati previsti metodi per recuperare i dati persi.
	\subsection{Lato client} % (fold)
	\label{sub:lato_client}
	Se cade il client, il server non dovrà fare nulla, ma il client dovrà salvarsi in qualche modo il suo identificativo e il punto nel quale è riuscito a compiere l'ultima istruzione in maniera completa sul server prima di cadere. \\
	Un approccio minimo viene effettuato tramite l'uso di una variabile statica nella classe \textbf{PuzzleSolverClient} denominata \textbf{STEP}.
	% subsection lato_client (end)
	
	\subsection{Lato server} % (fold)
	\label{sub:lato_server}
	Se cade il server, dovranno essere presenti dei meccanismi per salvare i dati elaborati completamente per i diversi client. \\
	Una volta salvati, il client, con la stessa variabile \textbf{STEP} citata in precedenza, potrà sapere fino a che punto il server ha effettuato il lavoro richiesto e se vuole potrà ripartire da la, una volta che il server sia tornato attivo. \\
	In questa versione del programma, se il server cade, viene solo stampato su terminale il punto dal quale è caduto.
	% subsection lato_server (end)

