% File: Packages.tex
% Created: 2014-11-07
% Author: Tesser Paolo
% Email: p.tesser921@gmail.com
% 
%
% Modification History
% Version	Modifier Date	Author			Change
% ====================================================================
% 0.0.1	2014-11-13	Tesser Paolo		inseriti package per la gestione ordinario del template
% ====================================================================
% 0.0.2	2014-11-13	Tesser Paolo		cambiato setup per il package hyperef, rimossa la linea: '\hypersetup{urlcolor = linkcolor}' che generava errori
%
%
%

\documentclass[11pt,a4paper]{article}

\usepackage{titlesec} % package per le sezioni a quattro indici
\usepackage[left=3.5cm,right=3.5cm,top=3.1cm,bottom=2.6cm]{geometry}
\usepackage[italian]{babel}
\usepackage[utf8]{inputenc}
\usepackage{lastpage}
\usepackage[T1]{fontenc}
\usepackage{enumitem} % package per la gestione degli elementi della lista

\usepackage{geometry}
\geometry{a4paper}

\usepackage{fancyhdr} % package di style per  l/c/rhead l/rfoot
\pagestyle{fancy}

\usepackage{graphicx} % package per la gestione delle immagini
\usepackage{color} % sempre per i colori

\usepackage{hyperref}
\hypersetup{colorlinks = true}
\hypersetup{urlcolor = blue}
\hypersetup{linkcolor = black}

\usepackage{longtable}
\usepackage{tabularx} % package meno complesso per la gestione delle tabelle

\usepackage{pdflscape} % package per mettere il foglio in orizzontale (utile in caso di figure che si estendono in orizzontale)
\usepackage{eurosym} % package per il simbolo del euro tramite comando \euro
