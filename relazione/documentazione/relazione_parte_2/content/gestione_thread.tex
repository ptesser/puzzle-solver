% File: gestione_thread.tex
% Created: 2014/01/13
% Author: Tesser Paolo
% Email: p.tesser921@gmail.com
% 
%
% Modification History
% Version	Modifier Date	Author			Change
% ====================================================================
% 0.0.1		2015/01/13		Tesser Paolo	inserita sezione capitolo
% ====================================================================
%

\section{Gestione dei Thread} % (fold)
\label{sec:gestione_dei_thread}
Nella seguente sezione vengono descritte le conseguenze che possono avvenire con l'avvio dei thread necessari alla risoluzione in parallelo del puzzle.

	\subsection{Numero Thread} % (fold)
	\label{sub:numero_thread}
	Si può sempre valutare il caso peggiore e sapere al massimo quanti thread ci saranno in esecuzione nelle diverse fasi dell'esecuzione dell'algoritmo. \\
	Di seguito verrà fornito il numero, illustrandolo in rapporto al flusso dell'algoritmo già descritto nella sezione \ref{sec:algoritmo_di_risoluzione_parallelo_}.
		\begin{enumerate}
			\item Durante la ricerca e la sistemazione del primo elemento del puzzle (quello in alto a sinistra) e quello in basso a sinistra TO DO;
			\item Durante l'ordinamento della colonna più a sinistra TO DO;
			\item Durante l'ordinamento di tutte le righe TO DO;
		\end{enumerate}
	% subsection numero_thread (end)
	
	\subsection{Interferenze, Deadlock, Attesa attiva} % (fold)
	\label{sub:interferenze_deadlock_attesa_attiva}
	TO DO
	% subsection interferenze_deadlock_attesa_attiva (end)
	
% section gestione_dei_thread (end)



