% File: note.tex
% Created: 2015-01-16
% Author: Tesser Paolo
% Email: p.tesser921@gmail.com
% 
%
% Modification History
% Version	Modifier Date	Author			Change
% ====================================================================
% 0.0.1	2015-01-16	Tesser-Paolo		inserita sezione e iniziata stesura
% ====================================================================
% 
%

\section{Note}
	\subsection{Versione JVM}
La versione di Java presente nella macchina utilizzata per l'implementazione del codice è quella 1.8.0\_20. \\
Non sono stati usati però costrutti particolari di questa versione e per l'utilizzo di API si è sempre fatto riferimento alla documentazione ufficiale della 1.7. \\
Per essere sicuri di rispettare la specifica che richiedeva al massimo la versione 1.7, si è testata la compilazione e l'esecuzione dell'applicativo sulle macchine di laboratorio dopo avere eseguito da terminale le istruzioni presenti al seguente link: \href{http://www.studenti.math.unipd.it/index.php?id=corsi#c620}{http://www.studenti.math.unipd.it} .

	\subsection{Compilazione}
Dalla root principale consegnata è possibile avviare il comando per la compilazione attraverso l'istruzione \textbf{make}. \\
Se si vuole lanciare il programma, testandolo con degli input definiti dal fornitore per provare l'applicativo, è possibile eseguire sempre da root il comando \textbf{make load} il quale lancerà il programma prendendo due file fissi come input e output. \\
Se si vuole invece lanciare il programma con dei file di test personalizzati è possibile eseguire il seguente script bash che compilerà automaticamente e lancerà l'applicativo: \\
\textbf{bash puzzlesolver.sh input output}, questo riceve i due file e lancia l'avvio del programma. \\ \\
\textbf{Attenzione}: lo script bash che lancia l'esecuzione del comando \emph{java} sul main principale del programma va ad eseguire il comando in un livello inferiori rispetto ad esso. Bisogna fare quindi attenzione al percorso del file in input che potrebbe generare un'eccezione qualora non fosse corretto.

