% File: cambiamenti.tex
% Created: 2014/01/13
% Author: Tesser Paolo
% Email: p.tesser921@gmail.com
% 
%
% Modification History
% Version	Modifier Date	Author			Change
% ====================================================================
% 0.0.1		2015/01/13		Tesser Paolo	inserita sezione capitolo
% ====================================================================
%

\section{Cambiamenti e Aggiunte}
In questa sezione verranno descritti i cambiamenti apportati alla precedente versione del programma per permettere all'algoritmo di risoluzione di essere parallelizzato. \\
In generale non vengono effettuate particolari rivisitazioni della precedente struttura, ma vengono introdotte nuove gerarchie ed estese alcune precedentemente create. \\
Il client rappresentato dalla classe \textbf{PuzzleSolver} andrà modificato solo nella scelta della politica di esecuzione dell'algoritmo, che da sequenziale passerà a parallela utilizzando la nuova classe descritta nella sezione \ref{ssub:package_solver_1}.

	\subsection{Estensioni} % (fold)
	\label{sub:estensioni}
		\subsubsection{Package solver} % (fold)
		\label{ssub:package_solver_1}
		Questo package contiene le classi che gestiscono la risoluzione del puzzle. \\
		Viene estesa la gerarchia che comprendeva alla base l'interfaccia \textbf{SolverStrategy} e come sottotipo la classe \textbf{SolverAlgStrategy}. \\
		In aggiunta a quanto già fatto viene estesa l'interfaccia anche con la nuova classe \textbf{SolverParStrategy} responsabile della risoluzione parallela del puzzle.
		TO DO (grafico)
		% subsubsection package_solver (end)
	% subsection estensioni (end)
	
	\subsection{Nuove gerarchie} % (fold)
	\label{sub:nuove_gerarchie}
		\subsubsection{Package solver} % (fold)
		\label{ssub:package_solver_2}
		Questo package contiene le classi che gestiscono la risoluzione del puzzle. \\
		In esso vengono aggiunte le classi che contengono l'attività logica
		TO DO (grafico)
		% subsubsection package_solver (end)
		
		\subsubsection{Package logger} % (fold)
		\label{ssub:package_logger}
		Questo package contiene solamente una classe che mi serve per la gestione di un file di log che viene usato per tracciare l'esecuzione del programma in alcuni punti. \\
		Essendo non importante ai fini delle richieste del proponente non ne verrà fornita una rappresentazione grafica.
		% subsubsection package_logger (end)
	% subsection nuove_gerarchie (end)

